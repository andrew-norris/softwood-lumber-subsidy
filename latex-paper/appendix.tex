\documentclass[12pt,a4paper]{article}

% Math and symbols
\usepackage{amsmath,amssymb,amsthm}

% Formatting
\usepackage[margin=1in]{geometry}
\usepackage{setspace}
\usepackage{parskip}

% Tables and floats
\usepackage{float}

% Hyperlinks
\usepackage[colorlinks=true,linkcolor=blue,citecolor=blue,urlcolor=blue]{hyperref}

\begin{document}

\doublespacing

\section{Appendix}
\subsection{Model}
\subsubsection{Model Definition}
\paragraph{Sawmill Firms}
\noindent\newline
\textbf{Table 1: Notation for Sawmill Firms}
\vspace{0.6em}

\noindent
\begin{tabular}{@{}llp{8cm}@{}}
\hline
\noalign{\vskip 0.2em}
\textbf{} & \textbf{Symbol} & \textbf{Description} \\
\hline
\noalign{\vskip 0.4em}
\textbf{Variables} 
& $Y^W$ & Lumber output \\
& $K$   & Capital input to sawmill firms \\
& $L$   & Labour input to sawmill firms \\[0.8em]

\textbf{Prices}
& $r$   & Rental rate of capital \\
& $w$   & Wage rate of labour \\
& $P^W$ & Price of lumber \\[0.8em]

\textbf{Parameters}
& $A^S$    & Technology parameter for sawmill firms \\
& $\alpha$ & Capital share parameter, $\alpha \in (0,1)$ \\
\hline
\end{tabular}

\vspace{1em}


\paragraph{Production Function:}
\begin{equation}\label{eq:sawmill_prod}
Y^W_t = A^S K_t^{\alpha} L_t^{1-\alpha}
\end{equation}

\paragraph{Budget Constraint:}
\begin{equation}\label{eq:sawmill_bc}
TC = r_t K_t + w_t L_t
\end{equation}

\paragraph{Construction Firms}
\noindent\newline
\textbf{Table 2: Notation for Construction Firms}
\vspace{0.6em}

\noindent
\begin{tabular}{@{}llp{8cm}@{}}
\hline
\noalign{\vskip 0.2em}
\textbf{} & \textbf{Symbol} & \textbf{Description} \\
\hline
\noalign{\vskip 0.4em}
\textbf{Variables}
& $Y^F$ & Final output produced by construction firms \\
& $W$   & Lumber input used in construction \\
& $\Psi$ & Alternative inputs to lumber \\[0.8em]

\textbf{Prices}
& $\mathcal{P}^W$ & Effective price of lumber \\
& $P^\Psi$ & Price of alternative inputs \\[0.8em]

\textbf{Parameters}
& $A^C$ & Technology parameter for construction firms \\
& $\theta$ & Input share or scaling parameter in construction production \\
& $\phi$ & Elasticity of substitution between lumber and alternatives \\
\hline
\end{tabular}

\vspace{1em}

\paragraph{Production}
\begin{equation}\label{eq:construction_prod}
Y^F_t = A^C\left(\theta\, W_t^{\phi} + (1-\theta)\, \Psi_t^{\phi}\right)^{1/\phi}.
\end{equation}

\paragraph{Cost}
\begin{equation}\label{eq:construction_cost}
TC^C_t = \mathcal{P}^{W}_t\, W_t + P^\Psi_t\, \Psi_t.
\end{equation}

\paragraph{Effective price of lumber}
\begin{equation}\label{eq:effective_lumber_price}
\mathcal{P}^{W}_t = (1-\omega)\, P^W_t.
\end{equation}

\paragraph{Government}
\noindent\newline
\textbf{Table 3: Notation for Government}
\vspace{0.6em}

\noindent
\begin{tabular}{@{}llp{8cm}@{}}
\hline
\noalign{\vskip 0.2em}
\textbf{} & \textbf{Symbol} & \textbf{Description} \\
\hline
\noalign{\vskip 0.4em}
\textbf{Variables}
& $G$ & Government expenditure on lumber tax credits \\[0.8em]

\textbf{Policy Instruments}
& $T$ & Lump-sum tax levied on households \\
& $\omega$ & Tax credit rate applied to lumber inputs \\
\hline
\end{tabular}

\vspace{1em}

\paragraph{Lumber tax credit expenditure}
\begin{equation}\label{eq:gov_spending}
G_t = \omega\, P^W_t\, W_t.
\end{equation}

\paragraph{Government budget constraint}
\begin{equation}\label{eq:gov_budget}
G_t = T_t.
\end{equation}

\paragraph{Households}
\noindent\newline
\textbf{Table 4: Notation for Households}
\vspace{0.6em}

\noindent
\begin{tabular}{@{}llp{8cm}@{}}
\hline
\noalign{\vskip 0.2em}
\textbf{} & \textbf{Symbol} & \textbf{Description} \\
\hline
\noalign{\vskip 0.4em}
\textbf{Variables}
& $C$ & Consumption of the final good \\
& $L$ & Labour supplied by the household \\[0.8em]

\textbf{Parameters}
& $\gamma$ & Preference weight on leisure in utility \\
\hline
\end{tabular}

\vspace{1em}

\paragraph{Preferences}
\begin{equation}\label{eq:hh_utility}
U = \log(C_t) + \gamma \log(1-L_t).
\end{equation}

\paragraph{Budget constraint}
\begin{equation}\label{eq:hh_budget}
C_t + K_{t+1} = w_t L_t + r_t K_t - T_t.
\end{equation}

\paragraph{External Demand}
\noindent\newline
\textbf{Table 5: Notation for External Demand}
\vspace{0.6em}

\noindent
\begin{tabular}{@{}llp{8cm}@{}}
\hline
\noalign{\vskip 0.2em}
\textbf{} & \textbf{Symbol} & \textbf{Description} \\
\hline
\noalign{\vskip 0.4em}
\textbf{Variables}
& $X$ & External demand for lumber \\ 
\hline
\end{tabular}

\vspace{1em}

\subsubsection{Market Clearing Conditions}
\begin{table}[H]
\centering
\caption{Notation for Market Clearing Conditions}
\vspace{0.6em}
\begin{tabular}{@{}llp{8cm}@{}}
\hline
\noalign{\vskip 0.2em}
\textbf{} & \textbf{Symbol} & \textbf{Description} \\
\hline
\noalign{\vskip 0.4em}
\textbf{Markets}
& $Y^W$ & Lumber market clearing output \\
& $Y^F$ & Final goods market clearing output \\
& $L$   & Labour market clearing quantity \\
\hline
\end{tabular}
\end{table}

Market clearing in each sector requires:
\begin{align}
Y^W &= W + X, \label{eq:lumber_market} \\
Y^F &= C,     \label{eq:final_market} \\
L   &= L^D.   \label{eq:labour_market}
\end{align}

\subsubsection{First Order Conditions}
\paragraph{Households}
\noindent\newline
The household chooses $\{C_t, L_t, K_{t+1}\}_{t=0}^{\infty}$ to maximize
\eqref{eq:hh_utility} subject to \eqref{eq:hh_budget}. The first-order conditions are:
\begin{align}
\frac{1}{C_t} &= \lambda_t, \label{eq:hh_foc_c} \\[0.4em]
\frac{\gamma}{1 - L_t} &= \lambda_t w_t, \label{eq:hh_foc_l} \\[0.4em]
\lambda_t &= \beta \lambda_{t+1}(r_{t+1} + 1 - \delta).
\label{eq:hh_foc_k}
\end{align}

These conditions imply the Euler equation:
\begin{equation}
\frac{1}{C_t}
= \beta \frac{1}{C_{t+1}} (r_{t+1} + 1 - \delta).
\end{equation}

And the Intratemporal Consumption Leisure Tradeoff
\begin{equation}
\frac{\gamma C_t}{1-L_t}
= w_t.
\end{equation}


\paragraph{Sawmill Firms}
\noindent\newline
Sawmill firms choose $K_t$ and $L_t$ to maximize profits given prices. The first-order conditions are:
\begin{align}
(1-\alpha) P^W_t A^S K_t^{\alpha} L_t^{-\alpha}
&= w_t, \label{eq:sawmill_foc_l} \\[0.4em]
\alpha P^W_t A^S K_t^{\alpha-1} L_t^{1-\alpha}
&= r_t. \label{eq:sawmill_foc_k}
\end{align}

\paragraph{Construction Firms}
\noindent\newline
Construction firms choose $W_t$ and $\Psi_t$ to minimize costs subject to
\eqref{eq:construction_prod}. The first-order conditions are:
\begin{align}
\theta A^C
\left(
\theta W_t^{\phi} + (1-\theta)\Psi_t^{\phi}
\right)^{\frac{1}{\phi}-1}
W_t^{\phi-1}
&= \mathcal{P}^W_t, \label{eq:const_foc_w} \\[0.6em]
(1-\theta) A^C
\left(
\theta W_t^{\phi} + (1-\theta)\Psi_t^{\phi}
\right)^{\frac{1}{\phi}-1}
\Psi_t^{\phi-1}
&= P^\Psi_t. \label{eq:const_foc_psi}
\end{align}

\subsection{Effective Tariff Rate}
\paragraph{Calculating the Rate}
\noindent\newline
The effective U.S. tariff rate on Canadian softwood lumber is constructed as a
firm-weighted average of countervailing and anti-dumping duties:
\begin{equation}
\tau_t = \sum_{i \in \mathcal{F}} \omega_i \tau_{i,t},
\end{equation}
where $\tau_{i,t}$ denotes the tariff rate applied to firm $i$ at time $t$, and
$\omega_i$ is firm $i$'s share of Canadian softwood lumber exports.

\paragraph{Export Weights}
\noindent\newline
\begin{table}[H]
\centering
\caption{Production Weights Used in the Construction of the Effective Tariff Rate}
\label{tab:tariff_weights}
\vspace{0.6em}
\begin{tabular}{@{}p{12cm}c@{}}
\hline
\noalign{\vskip 0.2em}
\textbf{Firm} & \textbf{Weight} \\
\hline
\noalign{\vskip 0.4em}
West Fraser Timber Co. Ltd. & 0.136 \\
Canfor Corporation & 0.111 \\
Resolute Forest Products & 0.096 \\
J.D. Irving, Limited & 0.055 \\
All remaining Canadian softwood lumber producers & 0.602 \\[0.6em]
\hline
\noalign{\vskip 0.2em}
\textbf{Total} & \textbf{1.000} \\
\hline
\end{tabular}
\end{table}

Tariff rates are calculated based on the dates of actions taken as part of the Annual Reviews (AR) 1 through 6. Constant weights were used for this paper based on 2023 export quantities. After AR3, resolute was not given individual tariff rates. After AR5 J.D. Irving was not given individual tariff rates.


\end{document}
